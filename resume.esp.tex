%-------------------------
% Curriculum Vitae en Latex
% Autor: Jake Gutierrez
% Basado en: https://github.com/sb2nov/resume
% Licencia: MIT
%------------------------

\documentclass[letterpaper,11pt]{article}

\usepackage{latexsym}
\usepackage[empty]{fullpage}
\usepackage{titlesec}
\usepackage{marvosym}
\usepackage[usenames,dvipsnames]{color}
\usepackage{verbatim}
\usepackage{enumitem}
\usepackage[hidelinks]{hyperref}
\usepackage{fancyhdr}
\usepackage[spanish]{babel}
\usepackage{tabularx}
\usepackage{fontawesome5}
\usepackage{multicol}
\setlength{\multicolsep}{-3.0pt}
\setlength{\columnsep}{-1pt}
\input{glyphtounicode}


%----------OPCIONES DE FUENTE----------
% sans-serif
% \usepackage[sfdefault]{FiraSans}
% \usepackage[sfdefault]{roboto}
% \usepackage[sfdefault]{noto-sans}
% \usepackage[default]{sourcesanspro}

% serif
% \usepackage{CormorantGaramond}
% \usepackage{charter}


\pagestyle{fancy}
\fancyhf{} % limpiar todos los campos
\fancyfoot{}
\renewcommand{\headrulewidth}{0pt}
\renewcommand{\footrulewidth}{0pt}

% Ajustar márgenes
\addtolength{\oddsidemargin}{-0.6in}
\addtolength{\evensidemargin}{-0.5in}
\addtolength{\textwidth}{1.19in}
\addtolength{\topmargin}{-.7in}
\addtolength{\textheight}{1.4in}

\urlstyle{same}

\raggedbottom
\raggedright
\setlength{\tabcolsep}{0in}

% Formato de secciones
\titleformat{\section}{
  \vspace{-4pt}\scshape\raggedright\large\bfseries
}{}{0em}{}[\color{black}\titlerule \vspace{-5pt}]

% Asegurar PDF legible por máquinas/ATS
\pdfgentounicode=1

%-------------------------
% Comandos personalizados
\newcommand{\resumeItem}[1]{
  \item\small{
    {#1 \vspace{-2pt}}
  }
}

\newcommand{\classesList}[4]{
    \item\small{
        {#1 #2 #3 #4 \vspace{-2pt}}
  }
}

\newcommand{\resumeSubheading}[4]{
  \vspace{-2pt}\item
    \begin{tabular*}{1.0\textwidth}[t]{l@{\extracolsep{\fill}}r}
      \textbf{#1} & \textbf{\small #2} \\
      \textit{\small#3} & \textit{\small #4} \\
    \end{tabular*}\vspace{-7pt}
}

\newcommand{\resumeSubSubheading}[2]{
    \item
    \begin{tabular*}{0.97\textwidth}{l@{\extracolsep{\fill}}r}
      \textit{\small#1} & \textit{\small #2} \\
    \end{tabular*}\vspace{-7pt}
}

\newcommand{\resumeProjectHeading}[2]{
    \item
    \begin{tabular*}{1.001\textwidth}{l@{\extracolsep{\fill}}r}
      \small#1 & \textbf{\small #2}\\
    \end{tabular*}\vspace{-7pt}
}

\newcommand{\resumeSubItem}[1]{\resumeItem{#1}\vspace{-4pt}}

\renewcommand\labelitemi{$\vcenter{\hbox{\tiny$\bullet$}}$}
\renewcommand\labelitemii{$\vcenter{\hbox{\tiny$\bullet$}}$}

\newcommand{\resumeSubHeadingListStart}{\begin{itemize}[leftmargin=0.0in, label={}]}
\newcommand{\resumeSubHeadingListEnd}{\end{itemize}}
\newcommand{\resumeItemListStart}{\begin{itemize}}
\newcommand{\resumeItemListEnd}{\end{itemize}\vspace{-5pt}}

%-------------------------------------------
%%%%%%  EL CURRICULUM COMIENZA AQUÍ  %%%%%%%%%%%%%%%%%%%%%%%%%%%%


\begin{document}

%----------ENCABEZADO----------
\begin{center}
    {\Huge \scshape Oscar Riquelme} \\ \vspace{6pt}
    {\Large Ingeniero Full Stack} \\ \vspace{4pt}
    \small \raisebox{-0.1\height}\faPhone\ +34601304460 ~ \href{mailto:x@gmail.com}{\raisebox{-0.2\height}\faEnvelope\ \underline{oscariquelmejato01@gmail.com}} ~ 
    \href{https://linkedin.com/in/oscar-riquelme-jato/}{\raisebox{-0.2\height}\faLinkedin\ \underline{linkedin.com/in/oscar-riquelme-jato}}  ~
    \href{https://github.com/}{\raisebox{-0.2\height}\faGithub\ \underline{github.com/oscariquelme01}}
    \vspace{-2pt}
\end{center}


%-----------EXPERIENCIA-----------
\section{\Large Experiencia}
  \resumeSubHeadingListStart

    % ORENES
    \resumeSubheading
      {Grupo Orenes}{Junio 2022 -- actualidad}
      {Desarrollador Full Stack Senior}{Madrid, España}
      \resumeItemListStart
        \resumeItem{Diseño de una arquitectura de microservicios que mejoró la escalabilidad y redujo significativamente la deuda técnica.}
        \resumeItem{Desarrollo de un servicio unificado para resolver funcionalidades de juegos, estandarizando a una única API en lugar de múltiples.}
        \resumeItem{Documentación del proyecto con OpenAPI y swaggerUI, creando un motor yaml custom para habilitar imports y live-reloading.}
        \resumeItem{Desarrollo de frontends para juegos usando Phaser3, HTML y CSS.}
        \resumeItem{Refactorización de la página de historial a una solución moderna con Vue.}
      \resumeItemListEnd

    
    % INFONIS
    \resumeSubheading
      {Infonis S.A.}{Feb 2023 -- Sept 2024}
      {Desarrollador Full Stack Freelance}{Madrid, España}
      \resumeItemListStart
        \resumeItem{Implementación de una herramienta interna para reemplazar SAP Crystal Reports.}
        \resumeItem{Configuración de SSO con Active Directory para estandarizar acceso a múltiples productos.}
        \resumeItem{Actualización de un proyecto masivo de React de componentes de clase a funcionales y actualización de dependencias.}
        \resumeItem{Refactorización de una herramienta para gestionar bases de datos y frontends CRM con personalización por cliente.}
    \resumeItemListEnd

    % DELONIA
    \resumeSubheading
      {Delonia Software}{Abril 2021 -- Junio 2021}
      {Desarrollador Frontend en Prácticas}{Madrid, España}
      \resumeItemListStart
      \resumeItem{Desarrollo de funcionalidades para aplicaciones con React y TailwindCss.}
      \resumeItem{Selección de tecnologías e implementación de una  solución de gestión de software interna.}
    \resumeItemListEnd
    
  \resumeSubHeadingListEnd
\vspace{-8pt}

%-----------PROYECTOS-----------
\section{\Large Proyectos}
    \vspace{-5pt}
    \resumeSubHeadingListStart
      \resumeProjectHeading
          {\textbf{Entorno Desarrollo Personal} $|$ \emph{Neovim, NixOS, Hyperland}}{Enero 2024}
          \resumeItemListStart
            \resumeItem{Configurción de Neovim para acelerar la edición e inspección de código.}
            \resumeItem{Creación de scripts de Hyperland y waybar para mejorar la navegación y estética del sistema operativo.}
            \resumeItem{Migración a NixOS para tener una configuración declarativa y compilaciones reproducibles con rollbacks.}
          \resumeItemListEnd
          \vspace{-13pt}

      \resumeProjectHeading
          {\textbf{Chainlink Hackathon -- Liquidez de Casino} $|$ \emph{Solidity, Phaser 3, React}}{Oct 2022}
          \resumeItemListStart
            \resumeItem{Desarrollo de un casino on-chain que permite a usuarios aportar liquidez mediante staking.}
            \resumeItem{Integración del servicio de Chainlink RNG para números aleatorios criptográficamente seguros.}
            \resumeItem{Lideré el equipo global de desarrolladores, definiendo el alcance del proyecto y asignando las tareas a cada desarrollador.}
            \resumeItem{Implementación de juegos de ruleta en Phaser 3 integrando con los smart contracts desarrollados.}
          \resumeItemListEnd 
          \vspace{-2pt}
    \resumeSubHeadingListEnd
\vspace{-8pt}


%-----------EDUCACIÓN-----------
\section{\Large Formación}
  \resumeSubHeadingListStart
    \resumeSubheading
      {Universidad Autónoma de Madrid}{Sept 2019 -- Junio 2023}
      {Grado en Ingeniería Informática}{Madrid, España}
  \resumeSubHeadingListEnd\vspace{-4pt}

%-----------HABILIDADES-----------
\section{\Large Habilidades Técnicas}
 \begin{itemize}[leftmargin=0.15in, label={}]
    \small{\item{
     \textbf{Lenguajes}{: Javascript/Typescript, Java, Python, C, HTML/CSS, SQL, Shell, Nix, Lua} \\
     \textbf{Herramientas}{: Neovim, NixOS/Linux, Git, Jira, Confluence, Bitbucket} \\
     \textbf{Tecnologías}{: React, Vue, Docker, Express, Redis, Redux, NextJS, OpenAPI, Phaser 3, FastApi, Flask} \\
    }}
     \textbf{Aprendiendo}{: Rust, Tokyo, Kubernetes, Self-Hosting, OpenCV} \\
 \end{itemize}
 \vspace{-10pt}


%-----------SOBRE MÍ---------------
 \section{\Large Sobre mí}
  Ingeniero informático apasionado por el desarrollo full stack y la tecnología en general. Mis principales intereses son el desarrollo web full stack, la arquitectura de software y los sistemas Unix. Actualmente estoy buscando un entorno para perfeccionar mis habilidades y construir proyectos web full stack innovadores.

\end{document}
